\documentclass[a4paper,12pt]{article}
\usepackage{extsizes}% Using font size
\usepackage[T2A]{fontenc}
\usepackage[utf8x]{inputenc} % supporting UTF8
\usepackage[russian,english]{babel}  % supporting Russian
\usepackage{listings}
\usepackage{graphicx} % for pictures
\usepackage{caption}% picture structure
\usepackage{subcaption}% picture structure
\usepackage{mmap}
\usepackage{booktabs}
\usepackage{indentfirst}% indent at the beginning of each paragraph
\usepackage{amsmath}
\usepackage[numbers,sort&compress]{natbib}% Nice citations [1,2,3,4] -> [1-4]
\DeclareMathOperator*\erf{erf}
\DeclareMathOperator*\erfc{erfc}
\DeclareMathOperator*\FT{FT}
\DeclareMathOperator*\IFT{IFT}
\DeclareMathOperator*\Col{Col}
\DeclareMathOperator*\Ew{Ew}
\DeclareMathOperator*\PME{PME}
\usepackage{float}
\restylefloat{table}
\usepackage{amssymb,amsmath,amsfonts,latexsym,mathtext,amsthm}% Мат. плюшки
\usepackage{multirow} % улучшенное форматирование таблиц
\usepackage{float}% Иначе картинки не работают
\usepackage[left=3cm,right=1.5cm,top=2.4cm,bottom=2.4cm]{geometry}% Рамки
\numberwithin{equation}{section}% Модная нумерация уравнений
\usepackage[onehalfspacing]{setspace}% Provides support for setting the spacing between lines in a document. Package options include singlespacing, onehalfspacing, and doublespacing.
\sloppy% по ширине..., иначе текст выходит за рамки
\addto\captionsenglish{
  \renewcommand{\contentsname}
    {Оглавление}
}

\addto\captionsenglish{
  \renewcommand{\figurename}
    {Рис.}
}
\addto\captionsenglish{
  \renewcommand{\tablename}
    {Табл.}
}
\addto\captionsenglish{
  \renewcommand{\lstlistingname}
    {Листинг}
}
\begin{document}

\begin{titlepage}
	\centering
	{Министерство образования и науки Российской Федерации\\
	Московский физико-технический институт (государственный университет)
	\\Факультет Аэрофизики и Космических Исследований
	\\Кафедра Информатики и Вычислительной Математики\par}
	\vspace{4 cm}
	{\scshape\LargeВыпускная квалификационная работа \par}
	\vspace{2 cm}
	{\scshape\Large***Название дипломной работы***\par}
	\vfill
	\raggedleft
	{Выполнила:\\
	студентка 53* группы\\
	\itshape Калмасова Азиза Онталаповна\par}
	{Научный руководитель:\\
	ассистент\\
	\itshape Казеннов Андрей Максимович\par}
	\vfill
	\centering
	{\large Москва\\
	2019\par}
\end{titlepage}

\tableofcontents
\newpage
\section{Введение}

***Тут будет введение ***

\newpage
\section{Новый заголовок}

*** Здесь будет еще какой-то текст ***

\subsection{Какой-то подпункт}

*** Текст какого-то подпункта ***

\subsubsection{Какой-то подпункт какого-то подпункта}

*** Текст какого-то подпункта какого-то подпункта ***

\newpage
\section{Реализация}

\section{Заключение}

\newpage
\bibliographystyle{gost780u}
\renewcommand{\refname}{Список литературы}
\addcontentsline{toc}{section}{Список литературы}
\begin{thebibliography}{99}


\end{thebibliography}
\end{document}
